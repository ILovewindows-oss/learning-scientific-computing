\subsection{Dimensionless numbers}

\paragraph{Knudsen}: Particles mean free path over system characteristic dimension. Division between rarefied gas (Boltzmann) and continuum mechanics (Navier-Stokes).

\paragraph{Prandtl}: Ratio of momentum diffusivity to thermal diffusivity $\mathrm{Pr}=\sfrac{\nu}{\alpha}$. High $\mathrm{Pr}$ indicates that momentum transfer is more effective than heat transfer (oils), while low values (liquid metals) indicate thermal boundary layer is more important than viscous one.

\paragraph{\href{https://en.wikipedia.org/wiki/Nusselt_number}{Nusselt}}: Ratio of convective to conductive heat transfer at a boundary in a fluid, defined as $\mathrm{Nu}=\sfrac{hL}{k}$. Often in buoyancy-driven flow analysis it is correlated as $\mathrm{Nu}=a\mathrm{Ra}^b$. A Nusselt number of value one represents heat transfer by pure conduction. Increasing this number implies a laminar conductive-dominant flow and then a convective dominant turbulent flow.

\paragraph{\href{https://en.wikipedia.org/wiki/Grashof_number}{Grashof}}: Ratio of buoyancy to viscous forces defined as $\mathrm{Gr}=\sfrac{g\beta(T_s-T_{\infty})L^3}{\nu^2}$ and is analogous to Reynolds number in natural convection. Increasing the value of this number above a given threshold promotes buoyancy driven flow. 

\paragraph{\href{https://en.wikipedia.org/wiki/Rayleigh_number}{Rayleigh}}: Product of Grashof $\mathrm{Gr}$ and Prandtl $\mathrm{Pr}$ numbers. Related to the transition from laminar to turbulent in buoyancy-driven flows. Laminar to turbulent is assumed to take place at $10^9$~\cite{Balaji2014}.

\endinput